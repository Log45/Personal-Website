\documentclass{article}
\usepackage[T1]{fontenc}

% Page Structure
\usepackage{geometry}
\geometry{
	top=.5in,
	left=1.25in,
	right=.5in,
	bottom=.5in,
}
\usepackage{changepage} % Will use later to adjust lengths

\newlength{\marginlen}
\newlength{\marginsep}
\newlength{\barlen}
\setlength{\marginlen}{0.75in}
\setlength{\marginsep}{0.05in}

\addtolength{\barlen}{\marginlen}
\addtolength{\barlen}{-\marginsep}


\usepackage{titlesec}
\titleformat{\section}%
	{\Large}% size
	{}% label format. empty
	{0pt}% space after label.
	{% This is the big bar
		\color{\accentcolor}%
		\hspace*{-\marginlen}%
		\raise0.25ex\hbox{\rule{\barlen}{1ex}}\hspace{\marginsep}%
	}
\setcounter{secnumdepth}{0} % turn off section numbering 
\titlespacing*{\section}{0pt}{1.0ex}{1.0ex} % keep 0, vertical space above, vertical space below.

\pagestyle{empty} % no page numbering


% Font
\usepackage{ebgaramond}
%\usepackage{libertine}
%\usepackage{ETbb}
%\usepackage{cochineal}


\usepackage{enumitem}
\setlist[itemize]{
	left=0pt,
	nosep,
	before=\small,
	after=\normalsize,
	topsep=2pt,
}

\usepackage{xcolor}
\usepackage{hyperref}
\hypersetup{colorlinks, allcolors=black}

\makeatletter
\newcommand{\email}[1]{\gdef\@email{#1}}
\newcommand{\linkedin}[1]{\gdef\@linkedin{#1}}
\newcommand{\github}[1]{\gdef\@github{#1}}

\renewcommand{\maketitle}{
    \begin{adjustwidth}{-\marginlen}{0pt}
        \begin{tabular}{@{} p{0.5\linewidth} @{} p{0.5\linewidth} @{} }
        {
            \fontsize{42}{72}\selectfont \@title
        }
        &    
        \hypersetup{allcolors=\accentcolor}
        \parbox{\linewidth}{%
            \raggedleft\large%
            \href{\@email}{\@email}\par%
            \href{\@linkedin}{\@linkedin}\par%
            \href{\@github}{\@github}%
        }
        \end{tabular}
    
        \bigskip
    \end{adjustwidth}
}

\makeatother

\setlength{\parindent}{0pt}


\newcommand{\accentcolor}{blue!50!black!100!white}

\title{Logan Endes}
\email{logan-endes@mail.rit.edu}
\linkedin{in/logan-endes}
\github{github.com/log45}

\reversemarginpar

\newcommand{\mpar}[1]{\leavevmode\marginpar{\raggedleft #1}}
\begin{document}
\maketitle

%\section{Objective}
%Motivated undergraduate student looking to apply for Computer Science and AI/ML Internships during the Summer of 2025 available to start after May 10th, 2025. 

\section{Experience}

\mpar{Jan 2024 --\par Aug 2024}
\textbf{AI Software Engineering Co-op}, \emph{Ecolab AI Accelerator Team}, Saint Paul, MN
\begin{itemize}

    \item Pioneered the use of \textbf{Computer Vision} AI to increase the efficiency of kitchens with an estimated \textbf{10x return on investment.}
    \item Maintained a \textbf{Large Language Model}-powered Copilot used by technicians and employees across the entirety of Ecolab. 
    \item Collaborated with engineers from Microsoft as part of Ecolab's new AI Accelerator Team.
    
    %\item Integrate AI solutions into Ecolab's system etc. 
\end{itemize}
\smallskip

\mpar{Aug 2023 --\par Dec 2023}
\textbf{Student Lab Instructor}, \emph{RIT Department of Computer Science}, Rochester, NY
\begin{itemize}
    \item Aided students through a lab section each week in learning Computer Science concepts and applying them to different problems using \textbf{Python}.
    \item Tutored students in \textbf{Python} and \textbf{Java} problems in the CS tutoring center each week. 
\end{itemize}
\smallskip

\mpar{June 2023 --\par Aug 2023}
\textbf{AI Research Intern}, \emph{Brookhaven National Laboratory}, Upton, NY
\begin{itemize}
        \item Researched the use of \textbf{prompt engineering Large Language Models} to make the process of researching medical isotopes more efficient using \textbf{PyTorch} and \textbf{Hugging Face's Transformers} library. 
        \item Found that models like LLAMA-2 were able to discern elements necessary in chemical separations, but that more research must be done for discerning specific isotopes needed.
	
\end{itemize}

% \smallskip

% \mpar{Mar 2022 --\par June 2022}
% \textbf{Volunteer}, \emph{"Blueberry" Research Project}, Rowan University
% \begin{itemize}
% 	\item Worked with Professor Nguyen of Rowan University to help develop a program that automates drones to take pictures of blueberry fields and estimate the yield.
% \end{itemize}

\section{Projects}

%\mpar{Date -- \par Present}
\textbf{Fencing Judge AI, \href{https://github.com/log45/Fencing-Judge-AI}{log45/Fencing-Judge-AI}}, \emph{Personal Project}
\begin{itemize}
	\item Researching the best \textbf{PyTorch} classification model for CV of live-video feed and training an AI model to accurately score Saber fencing bouts.
\end{itemize}

\smallskip

%\mpar{Nov 2021 - Present}
\textbf{AI Notetaker,
\href{https://github.com/log45/Notetaker}{log45/Notetaker}}, \emph{Personal Project}
\begin{itemize}
	\item Developing a web application to transcribe recordings into  and create notes with a large language model. 
\end{itemize}

\smallskip
\textbf{Keyboard Design, \href{https://github.com/log45/keyboard-design}{log45/keyboard-design},} \emph{Personal Project}
\begin{itemize}
    \item Designed a keyboard PCB using KiCad to be hot-swappable and programmable powered by a Raspberry Pi Pico.
\end{itemize}

%\smallskip
%\textbf{Alarm Clock Game, \href{https://github.com/log45/Alarm-Clock-Game}{log45/Alarm-Clock-Game},} \emph{Personal Project}
%\begin{itemize}
   % \item Developed a "Simon-Says" game that prompts the user to press highlighted squares in order to turn off an alarm.
%\end{itemize}


\section{Skills}

\mpar{Languages} Python, Java, LaTeX, C\#/ASP.NET, SQL, C/C++, Assembly
\smallskip

\mpar{Libraries} Tkinter, Transformers, PyTorch,  NumPy, OpenAI, Ultralytics, Supervision
\smallskip

%\mpar{Frameworks} React, Vue, Node.js	
%\smallskip

\mpar{Tools} VSCode, JetBrains IDEs, Anaconda, Git, KiCad
\smallskip

\mpar{Fluencies} English (Native), Japanese (Beginner)

\section{Education}
\textbf{Rochester Institute of Technology,} Rochester, NY\newline 
\textbf{B.S./M.S. Computer Science}\\
\textbf{B.S. Japanese Language and Culture}

Expected graduation: May 2026 B.S. / May 2027 M.S. \\
Dean's List: Fall 2022, Spring 2023, Fall 2023, Presidential Scholar, Performing Arts Scholar

\section{Extracurriculars \& Leadership}

\mpar{Aug 2022 -- \par Present}
\textbf{Member}, \emph{RIT Computer Science House}
\begin{itemize}
	\item Living in a learning community emphasizing hands-on learning and a strong social atmosphere while participating in member-led seminars and meeting a major project requirement.
\end{itemize}

\smallskip

\mpar{Aug 2022 -- \par Present}
\textbf{Member}, \emph{RIT Fencing Club}
\begin{itemize}
    \item Participating in weekly fencing practices and open-bouting sessions, fencing saber.
\end{itemize}

\smallskip

\mpar{June 2017 -- \par Jan 2022}
\textbf{Drum Major, Brass Captain, Member,} \emph{Pennsauken Apache Marching Band}
\begin{itemize}
    \item Developed skills in teamwork, discipline, and leadership while playing as part of a close-knit ensemble, leading rehearsals, and \\instructing students in marching technique.
\end{itemize}

\smallskip

\mpar{June 2021}
\textbf{State Assemblyman,} \emph{American Legion Jersey Boys State}
\begin{itemize}
    \item Simulated the creation of a 51st state with 500 student leaders from across New Jersey in order to solve problems in respective cities to learn the importance of discipline, respect, teamwork, and the political process.
\end{itemize}
\end{document}